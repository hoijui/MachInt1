\documentclass[a4paper,headings=small]{scrartcl}
\KOMAoptions{DIV=12}

\usepackage[utf8x]{inputenc}
\usepackage{amsmath}
\usepackage{graphicx}
\usepackage{multirow}

% instead of using indents to denote a new paragraph, we add space before it
\setlength{\parindent}{0pt}
\setlength{\parskip}{10pt plus 1pt minus 1pt}

\title{Machine Intelligence I - WS2011/2012\\Excercise 8}
\author{Robin Vobruba (343773)}
\date{\today}

\pdfinfo{%
  /Title    (Machine Intelligence I - WS2011/2012 - Excercise 8)
  /Author   (Robin Vobruba (343773))
  /Creator  ()
  /Producer ()
  /Subject  ()
  /Keywords ()
}



\begin{document}


\maketitle


\section{Excercise 8.1}

\subsection{8.1.a}
What is strict consistency?
Discuss its relationship with the empirical risk minimization frame-work.
\subsubsection{Answer}
Strict consistency says, that a read operation has to return the result
of the latest write operation which occurred on that data item.

\subsection{8.1.b}
Look up the term vc-dimension.
How is it related to linear separability?
\subsubsection{Answer}
vc-dimension is a measure of the capacity of a statistical classification
algorithm, defined as the cardinality of the largest set of points
that the algorithm can shatter (= separate).
it can predict a probabilistic upper bound on the test error of a classification model.
vc-dimension is a more general/abstract concept, which is not constrained
to _linear_ separability, but may be applied to any sort of classification model.


\section{Excercise 8.2}

see paper

\section{Excercise 8.3}
c&p von ex 6

\subsection{8.3.a}

\subsection{8.3.b}

\subsection{8.3.c}

\subsection{8.3.d}

\subsection{8.3.e}



\end{document}
